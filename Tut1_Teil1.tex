\documentclass{beamer}
\usepackage[ngerman]{babel}
\usepackage[utf8]{inputenc}
\usepackage{graphicx}
\usepackage{amssymb}
\usetheme{Warsaw}
\usecolortheme{default}

% Define user colors using the RGB model
\definecolor{hellblau}{rgb}{0.8,0.0,0.8}
\definecolor{dunkelblau}{rgb}{0.0,0.0,0.95}

\date{31.10.2014}
\author{ Richard Feistenauer }

\title {GBI Tutorium NR: 31}

\begin{document}

\begin{frame}
	\titlepage
\end{frame}

\begin{frame}
	\frametitle {Inhaltsverzeichnis}
	\tableofcontents
\end{frame}

\section{Organisatorisches}

\begin{frame}
	\frametitle {Organisatorisches}
	\begin{block}{Tutorium ist}
        		\begin{itemize}
			\item kurze Wiederholung der Vorlesung
			\item Anlaufstelle für Fragen
			\item Übungsbereich für aktuellem Vorlesungsstoff
			\item Ausgabestelle der Übungsblätter
			\item Freiwillig
		\end{itemize}
	\end{block}

	\begin{block}{Tutorium ist nicht}
        		\begin{itemize}
			\item Vorlesungs ersatz
			\item Lösungsstelle für kommendes Übungsblatt
		\end{itemize}
	\end{block}
\end{frame}

\begin{frame}
	\frametitle {Organisatorisches}
	\begin{block}{Übungsblatt}
        		\begin{itemize}
			\item Übungsblatt einzeln handschriftlich bearbeiten
			\item Abgabe Freitag 12:30 Uhr im Briefkasten im Keller
			\item Offensichtlich abgeschrieben $\Rightarrow$ 0 Punkte
			\item Ab Hälfte der Punkte bestanden (Voraussichtlich 120)
			\item Übungsschein zum Bestehen des Moduls notwendig
		\end{itemize}
	\end{block}
\end{frame}

\begin{frame}
	\frametitle {Organisatorisches}
	\begin{block}{Prüfung}
        		\begin{itemize}
			\item 4. März 2015 (14 Uhr)
			\item Nachprüfung im September. (Achtung Mathe Klausuren sind da 					auch!)
			\item Prüfung Notwendig für Orientierungsprüfung.
		\end{itemize}
	\end{block}
	
	\begin{block}{Kontakt / Information}
		\begin{itemize}
			\item gbi.tutorium@googlemail.com
			\item https://github.com/Richard-GBI/GBI-Folien
			\item http://gbi.ira.uka.de/
		\end{itemize}
	\end{block}
\end{frame}

\section{Alphabete}
\begin{frame}
	\frametitle{Alphabete}
	\begin{block}{Definition}
		Ein Alphabet ist eine \emph{endliche, nichtleere} Menge von Zeichen.
	\end{block}
	\begin{exampleblock}{Aufgaben}
		\begin{itemize}
			\item $\mathbb N$\(_{+}\) ?
			\item \(M=\{\phi,3,\psi,a\}\) ?
		\end{itemize}
	\end{exampleblock}
	\pause
	\begin{alertblock}{Notation}
		\begin{itemize}
			\item $\mathbb N$\(_{+}=\{1,2,3,\dots\}\) (positive ganze Zahlen)
			\item $\mathbb N$\(_{0}=\{0,1,2,3,\dots\}\) (nichtnegative ganze 					Zahlen)
		\end{itemize}
	\end{alertblock}
\end{frame}

\section{Aussagenlogik}
\begin{frame}
	\frametitle {Aussagenlogik}
	\begin{block}{}
		\begin{itemize}
			\item Eine Aussage ist ein Satz, der (objektiv) entweder wahr oder 					falsch sein kann
			\item Aussagen sind äquivalent (\(\Leftrightarrow\)), wenn sie die 					gleichen Wahrheitswerte besitzen
		\end{itemize}
	\end{block}
\end{frame}

\begin{frame}
	\frametitle {Aussagenlogik}
	\begin{block}{Logisches UND und ODER}
		\begin{center}
			\begin{tabular}{|c|c||c| }
				\hline
				A & B & A $\land$ B \\
				\hline
				wahr & wahr & \textbf{wahr}	\\
				\hline
				wahr & falsch & falsch \\
				\hline
				falsch & wahr & falsch \\
				\hline
				falsch & falsch & falsch \\
				\hline
			\end{tabular}
			\begin{tabular}{|c|c||c| }
				\hline
				A & B & A $\lor$ B \\
				\hline
				wahr & wahr & wahr \\
				\hline
				wahr & falsch & wahr \\
				\hline
				falsch & wahr & wahr \\
				\hline
				falsch & falsch & \textbf{falsch}	\\
				\hline
			\end{tabular}
		\end{center}
	\end{block}
	\pause	
	\begin{exampleblock}{Aufgabe}
		Stelle eine Wahrheitstabelle für den Ausdruck \((A\wedge B)\vee A\) auf.
	\end{exampleblock}
\end{frame}

\begin{frame}
	\frametitle {Implikation}
	\begin{center}
		\begin{tabular}{|c|c||c| }
			\hline
			A & B & $\Rightarrow$ \\
			\hline
			wahr & wahr & wahr \\
			\hline
			wahr & falsch & \textbf{falsch}	\\
			\hline
			falsch & wahr & wahr \\
			\hline
			falsch & falsch & wahr \\
			\hline
		\end{tabular}
	\end{center}
	\begin{alertblock}{Wichtig!}
		\begin{itemize}
			\item A $\Rightarrow$ B ist äquivalent zu \(\neg A\vee B\)
			\item D.h. man muss nur etwas tun, wenn A wahr ist. (Beweise)
		\end{itemize}
	\end{alertblock}
\end{frame}

\begin{frame}
	\frametitle {Implikation}
	\begin{center}
		\begin{tabular}{|c|c||c| }
			\hline
			A & B & $\Rightarrow$ \\
			\hline
			wahr & wahr & wahr \\
			\hline
			wahr & falsch & \textbf{falsch}	\\
			\hline
			falsch & wahr & wahr \\
			\hline
			falsch & falsch & wahr \\
			\hline
		\end{tabular}
	\end{center}
	\begin{exampleblock}{Aufgabe}
		Finde für F einen äquivalenten Ausdruck, in dem A und B jeweils höchstens 			einmal vorkommen.
		\begin{displaymath}
			F = (A\Rightarrow B) \Rightarrow ((B\Rightarrow A) \Rightarrow B)
		\end{displaymath}
	\end{exampleblock}
\end{frame}

\section{Relationen}
\subsection{Kartesisches Produkt}
\begin{frame}
	\frametitle {Kartesisches Produkt}
	\begin{block}{Definition}
		\(A\times B = \{(a,b)|a\in A \wedge b\in B\}\)\\
		Die Menge aller geordneten Paare (a,b) mit a aus A und b aus B
	\end{block}
	\pause
	\begin{exampleblock}{Aufgaben}
		\begin{itemize}
			\item Berechne \(\{a,b\} \times \{1,2,3\}\).\\
			\uncover<3->{ \(\{(a,1),(a,2),(a,3),(b,1),(b,2),(b,3)\}\)}
			\item Wieviele Elemente hat \(\{\alpha,\beta,\gamma,\delta\} \times 				\{42,43,44\}\)?\\
			\uncover<4->{12}
			\item Was ist \(\emptyset \times M\)?\\
			\uncover<5->{\(\emptyset\)}
		\end{itemize}
	\end{exampleblock}
\end{frame}

\subsection{Relationen}
\begin{frame}
	\frametitle{Relationen}
	\begin{block}{Definition}
		\begin{itemize}
			\item Eine Teilmenge \(R\subseteq A \times B\) heißt (binäre) Relation 			von A in B.
			\item Wenn A = B, spricht man von einer Relation auf der Menge A.
			\item Statt (a,b) $\in$ R kann man auch a R b schreiben bzw. statt \((a,b) \in R_{\geq}\) auch \(a \geq b\).
		\end{itemize}
	\end{block}
	\pause
	\begin{exampleblock}{Aufgabe}
		Wie ist die Kleiner-Gleich-Relation \(R_{\leq}\) auf der Menge M = 					\{1,2,3\} formell definiert?\\
		\pause
		\uncover<2->{\(R_{\leq}=\{(1,1),(1,2),(1,3),(2,2),(2,3),(3,3)\}\)}
	\end{exampleblock}
\end{frame}

\begin{frame}
	\frametitle {Eigenschaften von Relationen}
	\begin{tabular} {l l}
		\textbf {linkstotal} & eine Relation R $ \subseteq $ A x B ist linkstotal 	wenn gilt: \\
& $ \forall $ a $ \in $ A, $ \exists $ b $ \in $ B : ( a , b ) $ \in $ R \\
\\
		\textbf {rechtseindeutig} &eine Relation R $ \subseteq $ A x B ist rechtseindeutig wenn gilt: \\
& $ \forall $ a $ \in $ A, $ \forall $ b , c $ \in $ B : \\
& ( a , b ) $ \in $ R $ \land $ ( a , c ) $ \in $ R $\Rightarrow$ b = c \\
\\
		\textbf {rechtstotal } & eine Relation R $ \subseteq $ A x B ist rechtstotal wenn gilt: \\
& $ \forall $ b $ \in $ B, $ \exists $ a $ \in $ A : ( a , b ) $ \in $ R \\
\\
		\textbf {linkseindeutig} &eine Relation R $ \subseteq $ A x B ist linkseindeutig wenn gilt: \\
& $ \forall $ a , c $ \in $ A, $ \forall $ b $ \in $ B : \\
& ( a , b ) $ \in $ R $ \land $ ( c , b ) $ \in $ R $\Rightarrow$ a = c \\
	\end{tabular}
\end{frame}
		
\begin{frame}
	\frametitle {Eigenschaften von Relationen}
	\begin{tabular} {l l}
		\textbf {linkstotal} & Jedes Element aus A hat mindestens einen Partner\\ &in B \\
\\
		\textbf {rechtseindeutig} & Jedes Element aus A hat höchstens einen Partner\\ & in B \\ \\
\\
		\textbf {rechtstotal } & Jedes Element aus B hat mindestens einen Partner\\ & in A
\\
\\
		\textbf {linkseindeutig} & Jedes Element aus B hat höchstens einen Partner\\& in A
\\
	\end{tabular}
\end{frame}	

\begin{frame}
	\frametitle{Eigenschaften von Relationen}
	\begin{exampleblock}{Aufgaben}
		Sind folgende Relationen links-/rechtstotal, links-/rechtseindeutig?
		\begin{itemize}
			\item Die Gleichheitsrelation \(R_{=}\) auf $\mathbb R$
			\item Die Kleinerrelation \(R_{<}\) auf $\mathbb R$
		\end{itemize}
	\end{exampleblock}
\end{frame}

\subsection{Funktionen/Abbildungen}
\begin{frame}
	\frametitle{Funktionen/Abbildungen}
	\begin{block}{Definition}
		Eine Relation, die linkstotal und rechtseindeutig ist, nennt man Funktion 			oder Abbildung. \\
		Sei f: A $\rightarrow$ B eine Funktion. Dann ist:
		\begin{itemize}
			\item A der Definitionsbereich
			\item B der Zielbereich
			\item f(A) der Bildbereich von f
		\end{itemize}
	\end{block}
	\begin{exampleblock}{Aufgabe}
		Was bedeutet es wenn der Bildbereich gleich dem Zielbereich ist?
	\end{exampleblock}
	\end {frame}
	\begin {frame}
	\begin{block}{Eigenschaften von Funktionen/Abbildungen}
		\begin{itemize}
			\item linkseindeutig $\rightarrow$ injektiv
			\item rechtstotal $\rightarrow$ surjektiv
			\item injektiv + surjektiv = bijektiv
		\end{itemize}
	\end{block}
	\begin{exampleblock}{Aufgaben}
		Sind folgende Funktionen injektiv, surjektiv oder bijektiv?
		\begin{itemize}
			\item \(f:\mathbb R \rightarrow \mathbb R, x \mapsto x\)
			\item \(g:\mathbb N_{0} \rightarrow \mathbb N_{0}, x \mapsto 2x\)
		\end{itemize}
	\end{exampleblock}
\end{frame}

\section{Prädikatenlogik}
\begin{frame}
	\frametitle{Prädikatenlogik}
	Mit der Prädikatenlogik können wir viele Sachverhalte kurz und präzise 				darstellen.
	\pause
	\begin{block}{}
		Sei $W$ die Menge der möglichen Wetterformen und $S$ die Menge aller 				Studenten.\\
		$\heartsuit \subseteq$ $S \times W$ beschreibt "`Student liebt das 					Wetter"'
		\pause
		\begin{itemize}
			\item $\neg\exists s \in S$ : $\forall w \in W$ : $s \heartsuit w$\\
			\pause
			Es existiert kein Student, der alle Wetterformen liebt.
			\pause
			\item $\exists w \in W$ : $\forall s \in S$ : $s \heartsuit w$\\
			\pause
			Es existiert eine Wetterform, die jeder Student liebt.
			\pause
			\item $\forall s \in S$ : $\exists w \in W$ : $s \heartsuit w$\\
			\pause
			Für alle Studenten existiert eine Wetterform, die er liebt.
		\end{itemize}
	\end{block}
\end{frame}

\section{W\"orter}
\begin{frame}
	\frametitle{Wörter}
	\begin{block}{Vorbemerkung}
		\begin{itemize}
			\item $\mathbb Z_n$ = \pause \{ i $ \in $ $\mathbb N_0$ $ | $ 0 $ \le $ i $ \land $ i $ < $ n \}
			\item $\mathbb Z_0$ = \pause \{\}
		\end{itemize}
	\end{block}
	\begin{block}{In Worten}
		Wörter sind eine Surjektive Abbildung mit w: $\mathbb G_n$ $\rightarrow$ B $\subset$ A \\
		\begin{example}
			Das Wort w = hallo ist eine Abbildung \\
			w: $\mathbb Z_5$ $\rightarrow$ \{ a,h,l,o \} mit \\
			w(0) = h w(1) = a w(2) = l w(3) = l w(4) = o
		\end{example}
	\end{block}
\end{frame}

\subsection{Das leere Wort}
\begin{frame}
	\frametitle{Das leere Wort}
	\begin{block}{Das Wort}
		\begin{itemize}
			\item Das leere Wort wird mit dem $\epsilon$ dargestellt, und ist eine \\
			Abbildung von \{\} $\rightarrow$ \{\} \\
			\item \{\} x \{\} = \{\} \\
			\item $\epsilon$ hat die Länge 0 ist aber dennoch ein Element. \\
			\item wenn M = \{$\epsilon$\} dann ist M $\not=$ $\emptyset$ \\
			\item $|$M$|$ = 1
		\end{itemize}
	\end{block}
\end{frame}

\subsection{Konkatenation}
\begin{frame}
	\frametitle{Konkatenation von Wörtern}
	\begin{block}{Konkatenation von Wörtern}
		\begin{itemize}
			\item eine Konkatenation ist eine Verknüpfung mehrerer Zeichen(ketten) 			und wird als $\cdot$ dargestellt \\
			\item z.B. kann man hallo als h $\cdot$ a $\cdot$ l $\cdot$ l $\cdot$ 				o dargestellt werden.\\
			\item der Punkt ist allerding nicht notwendig, er kann wie das 						Malzeichen bei der Multiplikation weggelassen werden. \\
			\item mehrere Wörter können auch zu einem weiteren konkateniert 					werden.
		\end{itemize}
	\end{block}
\end{frame}

\section{Vollst\"andige Induktion}
\subsection[Einf\"uhrung]{Einf\"uhrung}
\begin{frame}
	\frametitle{Vollständige Induktion}
	\begin{block}{Was ist die vollst\"andige Induktion?}
		Eine oft benutzte sehr m\"achtige Beweistechnik
	\end{block}
	\begin{block}{Vorgehen?}
		\begin{enumerate}
			\item Die Behauptung f\"ur einen ersten Wert beweisen
			\item Annehmen dass die Behauptung f\"ur ``irgendeinen'' Wert gilt
			\item Behauptung ausgehend von dem bliebigen Wert f\"ur den n\"achsten 			Wert beweisen
		\end{enumerate}
	\end{block}
\end{frame}

\begin{frame}
	\frametitle{So sollte es aussehen}
	\begin{block}{Induktionsanfang}
		\begin{itemize}
			\item Beweis der Behauptung für einen (manchmal auch mehrere) 						''Startwerte''
		\end{itemize}
	\end{block}
	\pause
	\begin{block}{Induktionsannahme}
		\begin{itemize}
			\item F\"ur ein beliebiges aber festes x/k/n gelte: \ldots
			\item Wird im Induktionsschritt benutzt
		\end{itemize}
	\end{block}
	\pause
	\begin{block}{Induktionsschritt}
		\begin{itemize}
			\item Ausgehend von x die Behauptung f\"ur $x + 1$ beweisen
		\end{itemize}
	\end{block}
\end{frame}

\begin{frame}
	\frametitle{Ein erstes Beispiel}
	\begin{block}{Die Gaußsche Summenformel}
		\begin{center}
			$\sum\limits_{k=1}^{n}k = 1 + 2 + 3 + 4 + \ldots + n =
			\frac{n(n+1)}{2}$
		\end{center}
	\end{block}
\end{frame}

\begin{frame}
	\frametitle{Beweis!}
	\begin{block}{Induktionsanfang}
		\begin{center}
			$n = 1:$ \hspace{10mm} $\sum\limits_{k=1}^{n}k$ = $\sum\limits_{k=1}				^{1}k$ = 1
			= $\frac{1(1+1)}{2}$ = $\frac{n(n+1)}{2}$ \hspace{15mm} \uncover<2->{$				\Box$}
		\end{center}
	\end{block}
	\uncover<3-> 
	{
		\begin{block}{Induktionsvorraussetzung}
			\begin{center}
				F\"ur ein beliebiges aber festes n gelte: \\
				$\sum\limits_{k=1}^{n}k = \frac{n(n+1)}{2}$
			\end{center}
		\end{block} 
	}
	\uncover<4-> 
	{
		\begin{block}{Induktionsschluss}
			\begin{center}	
				$n = 1:$ \hspace{10mm}
				$\sum\limits_{k=1}^{n+1}k = (n+1)+ \sum\limits_{k=1}^{n}k 								\stackrel{\mathrm{I.V.}}=
				(n+1)+\frac{n(n+1)}{2} $ \\
				$= \frac{(n+1)(n+2)}{2} $ \uncover<5->{$\Box$}
			\end{center}
		\end{block} 
	}
\end{frame}

\subsection[Aufgaben]{Aufgaben}
\begin{frame}
	\frametitle{Jetzt seid ihr dran}
	\begin{block}{Eine Reihe}
		\begin{itemize}
			\item $a_{0} = 0$
			\item $a_{n+1} = a_{n} + 2n + 1$
		\end{itemize}
	\end{block}
	\vspace{5mm}
	\uncover<2->
	{
		\begin{block}{Zeige}
			$a_n = n^2$
		\end{block}
	}
\end{frame}

\begin{frame}
	\frametitle{Weiter gehts}
	\begin{block}{Noch ne Reihe}
		\begin{itemize}
			\item $a_{0} = 3$
			\item $a_{n+1} = a_{n} + 3$
		\end{itemize}
	\end{block}
	\vspace{5mm}
	\uncover<2->
	{
		\begin{block}{Zeige}
			\begin{itemize}
				\item Ideen?
				\visible<3->
				{
					\item $a_{n} = 3(n+1)$
				}
			\end{itemize}
		\end{block}
	}
\end{frame}

\begin{frame}
	\frametitle{Und jetzt mal was schweres}
	\begin{block}{Aufgabe }
		\begin{itemize}
			\item $x_0$ = 0 \\
			\item $\forall n \, \in \mathbb N_0 : x_{n+1} = x_n + (n+1)(n+2)$
			\item Tipp: $x_1, x_2, x_3, x_4$ ausrechnen
			\visible<2->
			{
				\item Wenn keine Idee: $ \frac{x(x+1)(x+2)}{3}$
			}
		\end{itemize}
	\end{block}
\end{frame}

\begin{frame}
	\frametitle{Unnützes Wissen}
	\begin{center}
		Anatidaephobia ist die Angst von einer Ente beobachtet zu werden.
	\end{center}
\end{frame}

\end{document}
