\documentclass{beamer}
\usepackage[ngerman]{babel}
\usepackage[utf8]{inputenc}
\usepackage{graphicx}
\usepackage{amssymb} 
\usepackage{amsmath}

\usepackage{hyperref}

\usetheme{Warsaw}
\usecolortheme{default}


\author{Richard Feistenauer}
\title{13. GBI-Tutorium von Tutorium Nr.31}
\date{6.Februar 2015}

\begin{document}

\begin {frame}
	\titlepage
\end {frame}

\begin {frame}
	\frametitle {Inhaltsverzeichnis}
	\tableofcontents
\end {frame}

\section{Äquivalenzrelationen}
\begin{frame}
	\frametitle{Eigenschaften}
	\begin{block}{}
		\begin{itemize}
			\pause
			\item Reflexivität ($\forall x : xRx$)
			\item Symmetrie ($\forall x,y : xRy \Rightarrow yRx$)
			\item Transitivität ($\forall x,y,z : xRy \land yRz \Rightarrow xRz$)
		\end{itemize}
	\end{block}
\end{frame}

\subsection{Kongruenz}
\begin{frame}
	\frametitle{Beispiel:Kongruenz modulo n}
	\begin{block}{}
		$xRy \Leftrightarrow (x - y)$ mod $n = 0$\\
		\pause
		\bigskip
		x und y sind genau dann äquivalent, wenn beide bei Division durch n den gleichen Rest liefern.\\
		\bigskip
		\begin{block} {Zeige, dass R Äquivalenzrelation ist.}
		\pause
		Reflexivität: x - x = 0\\
		Symmetrie: wenn x - y vielfaches von n dann auch y - x = -(x - y)\\
		Transitivität: x - y = $k_1$ * n und y - z = $k_2$ * n dann\\
		x - z = (x - y) - (y - z) = ($k_1 + k_2$) * n
		\end{block}
	\end{block}
\end{frame}

\subsection{Äquivalenzklassen}
\begin{frame}
	\frametitle{Äquivalentsklassen}
	\begin{itemize}
		\item Die Äquivalenzklasse von x $\in$ M ist $\{ y \in M : xRy \}$
		\item Schreibweise $[x]_R$ oder idR einfach $[x]$
		\item Die sog. Faktormenge von M nach R ist die Menge aller Äquivalenzklassen.
	\end{itemize}
\end{frame}

\begin{frame}
	\frametitle{Beispiel: Kongruenz modulo 3}
	\begin{itemize}
		\item Wie viele Äquivalenzklassen gibt es?
		\item Was sind diese Äquivalenzklassen, bzw wie schreibt man sie am besten auf?
	\end{itemize}
\end{frame}

\section{Halbordnung}
\subsection{Einführung}
\begin{frame}
	\frametitle{Antisymmetrie}
	\begin{itemize}
		\item R heißt antisymmetrisch, wenn für alle $x,y \in M$ gilt:\\
		$\> xRy \land yRx \Rightarrow x = y$
		\pause
		\item In Worten: R ist nur bei Gleichheit symmetrisch.
		\item Beispiel: Teilmengen-Relation:
		\item $A \subset B \land B \subset A \Rightarrow A = B$
	\end{itemize}
\end{frame}

\begin{frame}
	\frametitle{Definition}
	\begin{itemize}
		\item R heißt Halbordnung, wenn sie:\\
			\begin{itemize}
				\item reflexiv
				\item antisymmetrisch und
				\item transitiv ist.
			\end{itemize}
		\item Wenn R Halbordnung auf Menge M ist, nennt man M auch eine halbgeordnete Menge.	
		\item Darstellung häufig Hassediagramm (siehe Tafel).
	\end{itemize}
\end{frame}

\begin{frame}
	\frametitle{Beispiel}
	 Betrachte man die Realtion $\subset$ auf der Potenzmenge $2^M$ 
\end{frame}

\subsection{Besondere Elemente}
\begin{frame}
	\frametitle{Besondere Elemente}
	Sei $(M,\sqsubseteq)$ halbgeordnet und $T \subset M$.
	\begin{itemize}
		\item $x \in T$ heißt \textit{minimales Element von T}, wenn es kein $y \in T$ gibt, mit $y \sqsubseteq x$ und $y \neq x$
		\item $x \in T$ heißt \textit{kleinstes Element von T}, wenn für alle $y \in T$ gilt: $x \sqsubseteq y$
		\item $x \in T$ heißt \textit{maximales Element von T}, wenn es kein $y \in T$ gibt, mit $x \sqsubseteq y$ und $x \neq y$
		\item $x \in T$ heißt \textit{größtes Element von T}, wenn für alle $y \in T$ gilt: $y \sqsubseteq x$
		\pause
		\item Was sind die Unterschiede?
	\end{itemize}
\end{frame}


\begin{frame}
	\frametitle{Obere und Untere Schranke}
	Sei $(M,\sqsubseteq)$ halbgeordnet und $T \subset M$.
	\begin{itemize}
		\item $x \in M$ heißt \textit{obere Schranke von T}, wenn für alle $y \in T$ gilt: $y \sqsubseteq x$.
		\item $x \in M$ heißt \textit{untere Schranke von T}, wenn für alle $y \in T$ gilt: $x \sqsubseteq y$.
		\pause
		\item Beachte: untere und obere Schranken von T dürfen außerhalb von T liegen.
		\item Schranken müssen nicht existieren.
	\end{itemize}
\end{frame}

\begin{frame}
	\frametitle{Supremum / Infimum}
	\begin{itemize}
		\item Besitzt die Menge aller oberen Schranken einer Teilmenge T ein kleinstes Element, so heißt dies das Supremum von T (Schreibweise $\bigsqcup T$ oder sup(T)).
		\item Besitzt die Menge aller unteren Schranken einer Teilmenge T ein größtes Element, so heißt dies das Infimum von T.
		\item Müssen natürlich auch nicht existieren.
	\end{itemize}
\end{frame}

\subsection{Totale Ordnungen}
\begin{frame}
	\frametitle{Definition}
	\begin{itemize}
		\item R ist \textit{Ordnung} oder genauer \textit{totale Ordnung}, wenn gilt:
		\begin{itemize}
			\item R ist Halbordnung
			\item $\forall x,y \in M : xRy \lor yRx$		
		\end{itemize}
		\pause
		\item Es gibt keine unvergleichbaren Elemente.
		\item Beispiele?
	\end{itemize}
\end{frame}

\begin{frame}
	\frametitle{Zusatz}
	\begin{itemize}
		\item Es gibt noch einige andere Sachen zu Halbordnungen (vollständig, monotone und stetige Abbildungen, Fixpunktsatz).
		\item Wären noch mehr Definitionen gewesen, und ist normalerweise nicht Klausurrelevant.
		\item Schaut aber am besten zumindest mal über die Folien dazu drüber.
	\end{itemize}
\end{frame}

\begin {frame}
\frametitle {Ende}
	\begin {center}
		Noch Fragen?
	\end {center}
\end {frame}

\begin {frame}
\frametitle {Unnützes Wissen}
	\begin {center}
		Jedes mal wenn Beethoven komponierte, schüttete er sich etwas Eiswasser über den Kopf.
	\end {center}
\end {frame}

\end {document}