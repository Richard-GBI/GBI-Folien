\documentclass{beamer}
\usepackage[ngerman]{babel}
\usepackage[utf8]{inputenc}
\usepackage{graphicx}
\usepackage{amssymb}
\usepackage{amsmath}
\usepackage{hyperref}
\usetheme{Warsaw}
\usecolortheme{default}
\author{Richard Feistenauer}
\title{6.Tutorium von Tutorium Nr.31}
\date{5.Dezember 2014}

\begin{document}
\begin {frame}
	\titlepage
\end {frame}

\begin {frame}
	\frametitle {Inhaltsverzeichnis}
	\tableofcontents
\end {frame}

\section{Wiederholung}
\begin{frame}
	\frametitle{Letztes \"Ubungsblatt}
	\begin{block}{Infos zum letzten Blatt}
		\begin{itemize}
			\item LDC -1 (eigentlich gibts das nicht)
			\item add 00...1 (tut garantiert nicht das was ihr wollt)
		\end{itemize}
	\end{block}		
\end{frame}

\section{Kontextfreie Grammatiken}
\subsection[Definition]{Definition}
\begin{frame}
	\begin{block}{Grammatik Definition}
		\frametitle{Definition}
			Eine Grammatik ist ein Tupel
		\pause
		\begin{center}
			G = (N, T, S, P)
		\end{center}
	\end{block}
	\pause
	\begin{block}{ }
		N = Alphabet der \textbf{Nichtterminal}symbole \\
		T = Alphabet der \textbf{Terminal}symbole, N $\cap$ T 		= $\emptyset$ \\
		S = Startsymbol, S $\in$ N \\
		P = Produktionen, P $\subseteq$ N $\times$ (N $\cup$ 			T)*
	\end{block}
\end{frame}

\begin{frame}
	\frametitle{Beispiele f\"ur Grammatiken}
	\begin{block}{Einfache Grammatik}
		\begin{center}
			G = ( \{X\}, \{a, b\}, X,\{X $\rightarrow$ 
			$\epsilon \mid aX \mid$ bX\})
		\end{center}
		\pause
		\begin{itemize}
			\item Frage: \\
			\item Welche W\"orter lassen sich daraus 						ableiten? \\
			\item \uncover<3->{L(G) = \{a, b\}*}
		\end{itemize}
	\end{block}
\end{frame}

\subsection{Ableitung von Wörtern}
\begin{frame}
	\frametitle{Produktionen}
	\begin{itemize}
		\item Produktionen (oder auch Ableitungsregeln) 				schreiben wir in der Form n $\rightarrow$ w wobei n 
		$\in$ N und w $\in$ V* (V = N $\cup$ T)
		\item Auf der linken Seite von $\rightarrow$ steht 				genau ein Nichtterminalsymbol
		\item Auf der rechten Seite von $\rightarrow$ steht 			ein Wort aus V*
		\item Verschiedene Ableitungen aus dem gleichen 				Symbol können zusammengefasst werden: \\
		\{ S $\rightarrow$ a $\mid$ b $\mid$ $\epsilon$ \} ( 			= \{ S $\rightarrow$ a, S $\rightarrow$ b, S 
		$\rightarrow$ $\epsilon$ \} )
	\end{itemize}
\end{frame}

\begin{frame}
	\frametitle{Ableitung von W\"ortern}
	\begin{block} {Definition }
		\begin{itemize}
			\item $R_\Rightarrow$ ist eine Relation. $R_\Rightarrow\subseteq V^* \times V^*$
			\item Wir schreiben $w_1 \Rightarrow w_2$ wenn $w_2$ aus $w_1$ abgeleitet werden kann (durch anwenden 								\textbf{einer} Produktionsregel). \\
			Oder: $(w_1, w_2) \in R_\Rightarrow$
			\item Wir schreiben $w_1 \Rightarrow^* w_2$ wenn 					$w_2$ aus $w_1$ indirekt abgeleitet werden kann ( 					durch anwenden \textbf{mehrerer} Produktionsregeln 					nacheinander).
		\end{itemize}
	\end{block}
\end{frame}

\begin{frame}
	\frametitle{Ableitung von W\"ortern}
	\begin{example}{Beispiel}
		\begin{center}
			G = ( \{X\}, \{a, b\}, X,\{X $\rightarrow$ $\epsilon \mid aX \mid$ bX\})
		\end{center}
		X $\Rightarrow$ $\epsilon$ \\
		X $\Rightarrow$ bX $\Rightarrow$ baX $\Rightarrow$ baaX $\Rightarrow$ baa \\
		X $\Rightarrow$ aX $\Rightarrow$ abX $\Rightarrow$ abb $\Rightarrow$ abbaX $\Rightarrow$ abba \\
		Also: $\epsilon$, baa, abba $\in$ L(G)
	\end{example}
\end{frame}

\begin{frame}
\frametitle{Sprache einer Grammatik}
Eine Grammatik G beschreibt eine Sprache L(G):
\begin{center}
L(G) = \{w $\in$ T* $\mid$ w l\"asst sich aus S ableiten \} \\
L(G) = \{w $\in$ T* $\mid$ S $\Rightarrow$* w \}
\end{center}
W\"orter der Sprache besetehen nur aus Terminalsymbolen!
\end{frame}
\section{Aufgaben}
\begin{frame}
\begin{block}{Aufgabe}
Gib L(G) f\"ur die folgende Grammatik an:
\begin{center}
G = ( \{ A, B, C\}, \{a, b, c\}, A, P) \\
P = \{ A $\rightarrow$ B, \\
B $\rightarrow$ CC $\mid$a, \\
C $\rightarrow$ c
\}
\end{center}
\pause
\begin{example}
L(G) ) \{a, cc\}
\end{example}
\end{block}
\end{frame}
\begin{frame}
\begin{block}{Aufgabe}
$ A = \{a,b\}$\\
$ L = \{a^kb^ma^{m-k} : m,k \in \mathbb{N}_0 \land m \ge k\} $
\begin{itemize}
\item Gebe G an, sodass L(G) = L
\item Gebe Ableitungsbaum für aabbba an (mit G)
\item Gebe alle $n \in \mathbb{N}_0$ an, für die gilt $L \cap A^n \not= \{\}$
\item Sie $n$ so gewählt - wieviele Elemente enthält $L \cap A^n$
\end{itemize}
\end{block}
\end{frame}
\begin {frame}
\frametitle {Ende}
\begin {center}
Fragen?
\end {center}
\end {frame}
\begin{frame}
\frametitle{Unnützes Wissen}
\begin{center}
In Brasilien erzielte 2006 ein Balljunge ein offizielles Tor in einem Ligaspiel.
\end{center}
\end{frame}
\end {document}