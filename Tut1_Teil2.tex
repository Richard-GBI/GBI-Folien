\documentclass{beamer}
\usepackage[ngerman]{babel}
\usepackage[utf8]{inputenc}
\usepackage{graphicx}
\usepackage{amssymb}
\usetheme{Warsaw}
\usecolortheme{default}
\author{Richard Feistenauer}
\title{GBI-Tutorium 1}
\date{31.Oktober 2014}
\title {GBI Tutorium NR: 31}

\begin{document}
\begin{frame}
	\titlepage
\end{frame}

\begin{frame}
	\frametitle {Inhaltsverzeichnis}
	\tableofcontents
\end{frame}

\section{Formale Sprachen}
\subsection[Mengen]{Mengen}
\begin{frame}
	\frametitle{Mengenoperationen}
	\begin{block}{Definition}
		\begin{itemize}
			\item $ M_1 \cup M_2 = \{x \mid x \in M_1 \lor x \in M_2\} $ :Vereinigung
			\item $ M_1 \cap M_2 = \{ x \mid x \in M_1 \land x \in M_2 \} $ :Schnitt
			\item $ M_1 \setminus M_2 = \{x \mid x \in M_1 \land x \notin M_2 \}$ :Differenz
		\end{itemize}
	\end{block}	
\end{frame}

\begin{frame}
	\frametitle{Beispiele, Aufgaben}
	\begin{block}{Beispiele}
		\begin{itemize}[<+->]
			\item $\{1,2,3\} \cup \{2,3,4\} = \{1,2,3,4\}$ \newline 				\visible<2->{Kein Element kann in einer Menge ''doppelt'' vorkommen}
			\item $\{1,2,3\} \backslash \{2,3,4\} = \{1\}$
		\end{itemize}
	\end{block}
	\uncover<3->{
	\begin{block}{Aufgaben}
		\begin{itemize}
			\item $M \cup \{\} = ?$
			\item $M \cap \{\} = ?$
			\item $\{1,2,3\} \cap \{5,6,7\} = ?$
		\end{itemize}
	\end{block}}
\end{frame}

\subsection[Definition]{Definition}
\begin{frame}
	\frametitle{formale Definition}
	\begin{block}{Formale Sprachen}
		Eine formale Sprache L (Über einem Alphabet A) ist eine
		Teilmenge : $L \subseteq A$*
	\newline 
	\newline 
	\emph{Das heisst}: Eine Sprache über einem Alphabet ist eine
Teilmenge der Menge aller möglicher Wörter aus Zeichen
des Alphabetes
	\end{block}
	\begin{block}{Wichtig!}
		\begin{itemize}[<+->]
			\item $abb$ ist ein Wort
			\item $\{abb\}$ ist eine formale Sprache die nur aus dem Wort $abb$ besteht
			\item daraus folgt: \{$abb$\}* gibt es, $abb$* (noch) nicht
		\end{itemize}
	\end{block}
\end{frame}

\begin{frame}
	\frametitle{Beispiele}
	\begin{example}
		Sprache aller gültigen Java-Schlüsselwörter
		\begin{itemize}[<+->]
			\item Alphabet: $A = \{a,b,c\ldots,z\}$
			\item Sprache: $L = \{class, if, else, int, public, \ldots\}$
		\end{itemize}
	\end{example}
	\begin{example}
		\begin{itemize}[<+->]
			\item Alphabet: $A = \{a,b\}$
			\item Sei L die Sprache über A in denen das Teilwort ''$ab$'' nirgends vorkommt
			\item $L = \{a,b\}$*$\backslash\{w_{1}abw_{2}|w_{1},w_{2}\in\{a,b\}$*$\}$
			\item Vereinfacht:
			\visible<7->{\item $L = \{w_{1}w_{2}|w_{1} \in \{b\}$*$, w_{2} \in
\{a\}$*$\}$}
		\end{itemize}
	\end{example}
\end{frame}

\section{Konkatenation}
\subsection[Konkatenation formaler Sprachen]{Konkatenation formaler Sprachen}
\begin{frame}
	\frametitle{Produkt von Sprachen}
	\begin{block}{Definition}
		$L_{1}\cdot L_{2} = \{w_{1}w_{2}|w_{1} \in L_{1} \land w_{2} \in L_{2}\}$
	\end{block}
	\begin{example}
		Von gerade eben: formale Sprache aller Wörter über $A=\{a,b\}$ in denen das
Teilwort $"ab"$ nirgends vorkommt: \newline
Kann man jetzt auch so schreiben: $L = \{b\}$*$\{a\}$*
	\end{example}
\end{frame}

\begin{frame}
	\frametitle{Beispiele}
	\begin{example}
		\begin{itemize}[<+->]
			\item Sei $L_{1} = \{KARTOFFEL, NUDEL\}$
			\item Sei $L_{2} = \{SALAT, AUFLAUF\}$
			\item Was ist $L_{1} \cdot L_{2} ?$
			\visible<4->{\{KARTOFFELSALAT, NUDELSALAT, KARTOFFELAUFLAUF, NUDELAUFLAUF\}}
		\end{itemize}
	\end{example}
	\begin{example}
		\visible<5->{Formale Sprache aller legalen Ganzzahlen:}
		\begin{itemize}[<6->]
			\item Alphabet: $A = \{0,1,\ldots,9\}$
			\item $L_{G} = A \cdot A$*
			\item Was fehlt?
			\visible<7->{\item die negativen Zahlen!}
			\visible<8->{\item besser: $L_{G} = \{\epsilon,-\} \cdot A \cdot A$*}
		\end{itemize}
	\end{example}
\end{frame}

\begin{frame}
	\frametitle{Ein letztes Beispiel}
	\begin{itemize}[<+->]
		\item Sei $L_{1} = \{a^{n}|n \in \mathbb N_{0}\}$
		\item Sei $L_{2} = \{b^{n}|n \in \mathbb N_{0}\}$
		\item $L_{1} \cdot L_{2} = ?$
		\item $ L_1L_2 = \{ a^{k}b^{m} \mid k \in \mathbb N_{0} \land m \in \mathbb N_{0} \} = \{ a\}^{*}\{ b\}^{*}$
	\end{itemize}
\end{frame}

\section{Konkatenationsabschluss}
\subsection[Der Konkatenationsabschluss]{Der Konkatenationsabschluss}
\begin{frame}
	\frametitle{Potenzen von Sprachen}
	\begin{block}{Definition}
		\begin{itemize}
			\item $L^{0} = \{\epsilon\}$
			\item $L^{i+1} = L^{i} \cdot L$
		\end{itemize}
	\end{block}
	\begin{example}
		Sei $L = \{a\}$*$\{b\}$*
		\begin{itemize}[<+->]
			\item $L^{0} = \{\epsilon\}$
			\item $L^{1} = \{\epsilon,a,b,aa,ab,bb,aaa\ldots\}$
			\item $L^{2} = \{\epsilon,aabbbaaaaabb,aaabbab,aaaaa,bbbbbb\ldots\}$
			\item usw.
		\end{itemize}
	\end{example}
\end{frame}

\begin{frame}
	\frametitle{Der Konkatenationsabschluss}
	\begin{block}{Definition}
		\begin{itemize}[<+->]
			\item $L^{+} = \bigcup\limits_{i = 1}^{\infty} L^{i}$
			\item $L^{*} = \bigcup\limits_{i = 0}^{\infty} L^{i}$
		\end{itemize}
	\end{block}
	\begin{example}
		\begin{itemize}[<+->]
			\item $L = \{a\}$*$\{b\}$*
			\item $L^{*}$ ? \visible<5->{ = $\{a,b\}$*}
			\visible<6->{\item ``Beweis'': Zerhacke beliebiges aber festes $w \in
\{a,b\}$* an allen Stellen an denen auf ein b ein a folgt. Die entstehenden
Teilworte sind aus $L$}
		\end{itemize}
	\end{example}
\end{frame}

\section{Beweisführung}
\subsection{Beweise}
\begin{frame}
	\frametitle{Übung}
	\begin{block}{Beweise}
		Beweise: $L^{*} \cdot L = L^{+}$
	\end{block}
	\begin{block}{Hinweis}
		Seien A und B zwei Mengen
		\newline $A \subseteq B \ \land \ B \subseteq A \Rightarrow A = B$
	\end{block}
\end{frame}

\begin{frame}
	\frametitle{Beweis}
	\begin{block}{$L^{*} \cdot L \subseteq L^{+}$}
		Wenn $w \in L^{*} \cdot L$, dann $w = w_{1}w_{2}$ mit $w_{1} \in L^{*}$ und
$w_{2} \in L$ \newline
		Also existiert ein $i \in \mathbb N_{0}$ mit $w_{1} \in L^{i}$ \newline
Also $w = w_{1}w_{2} \in L^{i} \cdot L = L^{i+1}$ \newline
Da $i+1 \in \mathbb N_{+}$, ist $L^{i+1} \subseteq L^{+}$, also $w \in L^{+}$
	\end{block}
	\begin{block}{$L^{*} \cdot L \supseteq L^{+}$}
		Wenn $w \in L^{+}$, dann existiert ein $i \in \mathbb N_{+}$ mit $w \in L^{i}$
\newline 
		Da $i \in \mathbb N_{+}$ ist $i = j + 1$ für ein $j \in \mathbb N_{0}$
\newline 
		Also ist für ein $j \in \mathbb N_{0}$: $w \in L^{j+1} = L^{j} \cdot L$
\newline 
		Also $w = w_{1}w_{2}$ mit $w_{1} \in L^{j}$ und $w_{2} \in L$ \newline
Wegen $L^{j} \subseteq L^{*}$ ist $w = w_{1}w_{2} \in L^{*} \cdot L$
	\end{block}
\end{frame}

\begin{frame}
	\frametitle{Zusammenfassung}
	\begin{block}{2 Sprachen}
		\begin{itemize}
			\item $L_{1} = \{a^{k}b^{m}|k,m \in \mathbb N_{0} \land \ k$ mod 2 = 0 $\land
\ m$ mod 3 = 1\}
			\item $L_{2} = \{a^{k}b^{m}|k,m \in \mathbb N_{0} \land \ k$ mod 2 = 1 $\land
\ m$ mod 3 = 0\}
		\end{itemize}
	\end{block}
	\begin{block}{Aufgabe}
		Was ist?
		\begin{itemize}
			\item $ L = L_1 \cdot L_2$
			\item $ L = L_1 \cap L_2$
		\end{itemize}
	\end{block}
\end{frame}

\begin{frame}
	\frametitle {Ende}
	\begin{center}
		Fragen?! \\
	\end{center}
\end{frame}

\begin{frame}
	\frametitle{Unnützes Wissen}
	\begin{block}{}
		\begin{center}
			Folgendes Gesetz gilt in Texas: Wenn sich zwei Züge an 					einem Bahnübergang begegnen, müssen beide Züge halten. 					Jeder der beiden Züge muss so lange stehen bleiben, bis 				der jeweils andere vorbeigefahren ist.
		\end{center}
	\end{block}{}
\end{frame}
\end{document}