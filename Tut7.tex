\documentclass{beamer}
\usepackage[ngerman]{babel}
\usepackage[utf8]{inputenc}
\usepackage{graphicx}
\usepackage{amssymb} 
\usepackage{amsmath}

\usepackage{hyperref}

\usetheme{Warsaw}
\usecolortheme{default}


\author{Richard Feistenauer}
\title{7.GBI-Tutorium von Tutorium Nr 31}
\date{12.Dezember 2014}

\begin{document}

\begin {frame}
	\titlepage
\end {frame}

\begin {frame}
	\frametitle {Inhaltsverzeichnis}
	\tableofcontents
\end {frame}

\section{Wiederholung}

\begin{frame}
	\frametitle{Letztes \"Ubungsblatt}
		\begin{itemize}
			\item Huffman-Codierung größtenteils gut
			\item bei 4 am besten mal Musterlösunganschaun
			\item R = (a,b) ?
			\item Relationen != Funktionen ( R(x) geht nicht!)
		\end{itemize}
\end{frame}

\section{Graphen} 
\subsection{Motivation}

\begin{frame}
	\frametitle{Motivation}

	\begin{block}{Ein paar Gedankenspiele}

		\begin{itemize}
			\item Wie w\"urde man ein Stra{\ss}ensystem modellieren ?
			\item Welche Bedingungen m\"ussten erf\"ullt sein, was w\"are ''nice to have'' ?
		\end{itemize}


	\end{block}
\end{frame}

\subsection{Definition}

\begin{frame}
	\frametitle{Definition}
	\begin{block}{Definition}
		Ein gerichteter Graph ist ein Tupel G = (V, E) wobei
		\begin{itemize}
			\item V = eine nichtleere, endliche Menge, genannt Knoten (vertex)
			\item E $\subseteq$ V x V = eine Relation in den Knoten, gegannt Kanten (edges)
		\end{itemize}	
	\end{block}
	\begin{block}{Definition}
		Eine Kante (x, x) $\in$ E hei{\ss}t Schlinge
	\end{block}
\end{frame}

\begin{frame}
	\frametitle{Definition}
	\begin{block}{Definition }
		Graphen werden typischerweise graphisch dargestellt.
		\begin{itemize}
			\item Die Knoten als Punkte oder Kreise
			\item Die Kanten (x, y) als Pfeile von Punkt x nach Punkt y
		\end{itemize}
	\end{block}

	\begin{example}
		Einfaches Beispiel an der Tafel
	\end{example}
\end{frame}

\begin{frame}
	\frametitle{Aufgaben}
	\begin{block}{Aufgaben }
		\begin{itemize}
			\item Wieviele Kanten kann ein Graph mit Schlingen maximal haben ?
			\item Wieviele ohne Schlingen ?
		\end{itemize}
	\end{block}
\end{frame}

\begin{frame}
	\frametitle{Teilgraphen}
	\begin{block}{Definition}
		 G' = (V', E') ist ein Teilgraph von G = (V, E) wenn:
		\begin{itemize}
			\item V' $\subseteq$ V
			\item E' $\subseteq$ E $\cap$ V' x V'
			\item Was hei{\ss}t das?
			\item Beispiel an der Tafel
		\end{itemize}	
	\end{block}
\end{frame}

\section{Isomorphie von Graphen}
\begin{frame}
	\frametitle{Isomorphie von Graphen}
	\begin{block}{Definition}
		Zwei Graphen sind isomorph, wenn sie sich nur in den Benennungen der Knoten unterscheiden.\\
		$\Rightarrow$ Ein Graph lässt sich so umbenennen, dass er mit dem anderen identisch ist.
	\end{block}
\end{frame}

\section{Pfade und Erreichbarkeit}

\begin{frame}
	\frametitle{Definition von Pfaden}
	\begin{block}{Weg von Knoten x nach Knoten y}
		Aber
		\begin{itemize}
			\item Wenn ein Weg von x nach y existiert, muss kein Weg von y nach x f\"uhren
			\item Kante muss in die richtige Richtung weisen
			\item Knoten d\"urfen in Pfaden mehrfach vorkommen (sollten es aber nicht)
			\item Wege k\"onnen auch unterschiedlich lang sein.
		\end{itemize}
	\end{block}
\end{frame}

\section{Ungerichtete Graphen}

\begin{frame}
	\frametitle{Ungerichtete Graphen}
	\begin{block}{Zu beachten}
		\begin{itemize}
			\item f\"ur x $\ne$ y ist \{x, y\} eine ZWEIelementige Menge, ohne eine Festlegung der Reihenfolge
			\item f\"ur x = y ist \{x, y\} = \{x\} eine EINelementige Menge
		\end{itemize}
	\end{block}
	\begin{example}{ }
		Wie ist das mit der Anzahl Kanten eines ungerichteten Graphen mit n Knoten ? 
	
	\end{example}
\end{frame}

\section{Aufgaben}

\begin{frame}
	\frametitle{Aufgaben}
	\begin{block} {Aufgabe}
		Gegeben sei der Graph G = (V, E) mit V = $\{0, 1\}^3$ und \\
		E = \{( xw, wy) $\mid x, y \in \{0, 1\} \land w \in \{0, 1\}^2$
		\begin{itemize}
			\item Zeichnen Sie den Graphen
			\item Geben Sie einen Zyklus in G an, der au{\ss}er dem Anfangs- und Endknoten jeden Knoten von G genau 					einmal enth\"alt.
			\item Geben Sie einen geschlossenen Pfand in G an, der jede Kante von G genau einmal enth\"alt.
		\end{itemize}
	\end{block}
\end{frame}

\begin{frame}
	\frametitle{Unnützes Wissen}
	\begin{center}
		Es gibt einen Wurm, der unter dem Augenlid eines Nilpferds lebt, und sich von dessen Tränen ernährt.
	\end{center}
\end{frame}

\end {document}
