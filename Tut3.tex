\documentclass{beamer}
\usepackage[ngerman]{babel}
\usepackage[utf8]{inputenc}
\usepackage{graphicx}
\usepackage{amssymb}
\usepackage{amsmath}
\usepackage{hyperref}
\usetheme{Warsaw}
\usecolortheme{default}

\author{Richard Feistenauer}
\title{GBI-Tutorium 3}
\date{14.November.2014}

\begin{document}
\begin{frame}
	\titlepage
\end{frame}

\begin{frame}
	\frametitle {Inhaltsverzeichnis}
	\tableofcontents
\end{frame}

\section{Wiederholung}
\subsection{Übungsblatt}
\begin{frame}
	\frametitle{Letztes \"Ubungsblatt}
	\begin{block}{Probleme}
		\begin{itemize}
			\item Formale Definitionen
			\item Vollständige Induktion
			\item induktiv $\approx$ rekursiv
		\end{itemize}
	\end{block}
\end{frame}

\subsection{Huffmancode}
\begin{frame}
	\frametitle{Huffman \"Ubung}
	\begin{block}{}
		was ist die kürzeste Codierung für das Wort $"$MISSISSIPPI$"$ \\
		$-$ ohne Blockcode\\
		$-$ mit Blockcode	
	\end{block}
\end{frame}

\section{Speicher}
\subsection{Bit und Byte}
\begin{frame}
	\frametitle{Bit und Byte}
	\begin{block}{Definition}
		Ein Bit ist ein Zeichen des Alphabetes \{0,1\}\\
		Ein Byte sind üblicherweise 8 Bit. \{vor allem früher war das 		nicht immer so\} \\
		Deshalb gibt es die Bezeichnung Octet das ist ein Byte mit 8 			Bit
	\end{block}
\end{frame}

\subsection{Größenpr\"afixe}
\begin{frame}
	\frametitle{Größenpr\"afixe Dezimal}
	\begin{center}
  		\begin{tabular}{*{6}{c}}
  			\hline
   			$10^{-3}$ & $10^{-6}$ & $10^{-9}$ & $10^{-12}$ & 						$10^{-15}$ & $10^{-18}$ \\
    		$1000^{-1}$ & $1000^{-2}$ & $1000^{-3}$ & $1000^{-4}$ 					& $1000^{-5}$ & $1000^{-6}$ \\
    		milli & mikro & nano & pico & femto & atto \\
    		m & $\mu$ & n & p & f & a \\
    		\hline
    		$10^3$ & $10^6$ & $10^9$ & $10^{12}$ & $10^{15}$ & 						$10^{18}$ \\
    		$1000^1$ & $1000^2$ & $1000^3$ & $1000^4$ & $1000^5$ 					& $1000^6$ \\
    		kilo & mega & giga & tera & peta & exa \\
   			k & M & G & T & P & E \\
   			\hline
  		\end{tabular}
	\end{center}
\end{frame}

\begin{frame}
	\frametitle{Größenpr\"afixe Bin\"ar}
	\begin{center}
  		\begin{tabular}{*{6}{c}}
   			\hline
    		$2^{10}$ & $2^{20}$ & $2^{30}$ & $2^{40}$ & $2^{50}$ & 					$2^{60}$ \\
    		$1024^1$ & $1024^2$ & $1024^3$ & $1024^4$ & $1024^5$ & 					$1024^6$ \\
    		kibi & mebi & gibi & tebi & pebi & exbi \\
    		Ki	& Mi & Gi & Ti & Pi & Ei\\
    		\hline
  		\end{tabular}
	\end{center}
	Eine Terrabyte Festplatte hat für gewöhnlich nur 1000000000000 			Bytes also ca. 930 Gigabytes
\end{frame}

\subsection{Speicher}
\begin{frame}
	\frametitle{Speicher als Tabelle}
	\begin{center}
		\begin{tabular}{{c}{c}}
  		\begin{tabular}{{c}|{c}}
  			\hline
   			Adresse 1 & Wert 1\\
    		Adresse 2 & Wert 2\\
    		Adresse 3 & Wert 3\\
   			... & ... \\
   			Adresse n & Wert n\\
   			\hline
  		\end{tabular}
  		&
  		\begin{tabular}{{c}|{c}}
  			\hline
   			000 & 01001011\\
    		001 & 10010101\\
    		010 & 00111001\\
   			... & ... \\
   			111 & 11100101\\
   			\hline
  		\end{tabular}			
		\end{tabular}
	\end{center}
\end{frame}

\begin{frame}
	\frametitle{Speicher als Abbildung}
	\begin{block} {Abbildung}
		m: Adr $\rightarrow$ Val\\
		m(a) = v\\
		~\\
		memread : Mem x Adr $\rightarrow$ Val\\
		memread : $Val^{Adr}$ x Adr $\rightarrow$ Val\\
		memread(m, a) $\mapsto$ m(a)\\
		~\\
		memwrite : Mem x Adr x Val $\rightarrow$ Mem\\
		memwrite : $Val^{Adr}$ x Adr x Val $\rightarrow Val^{Adr}$\\
		memwrite(m, a, v) $\mapsto$ m'
	\end{block}
\end{frame}

\begin{frame}
	\frametitle{\"Ubung}
	\begin{block} {}
		\begin{itemize}[<+->]
			\item memread(memwrite(m,a,v),a) = \visible<2-> {v}
			\item memread(memwrite(m,a',v'),a) = \visible<3-> {memread(m,a)}
		\end{itemize}
	\end{block}
\end{frame}

\begin{frame}
	\begin{center}
		Fragen?
	\end{center}
\end{frame}

\begin{frame}
	\frametitle{Unnützes Wissen}
	\begin{center}
		Als Bibliotaphen  bezeichnet man Menschen, die ihre Bücher in 		Verstecken aufbewahren und nicht verleihen.
	\end{center}
\end{frame}
\end{document}