\documentclass{beamer}
\usepackage[ngerman]{babel}
\usepackage[utf8]{inputenc}
\usepackage{graphicx}
\usepackage{amssymb}
\usepackage{amsmath}
\usepackage{hyperref}
\usetheme{Warsaw}
\usecolortheme{default}

\author{Richard Feistenauer}
\title{4.GBI-Tutorium von Tutorium Nr. 31}
\date{21.November.2014}

\begin{document}
\begin{frame}
	\titlepage
\end{frame}

\begin{frame}
	\frametitle {Inhaltsverzeichnis}
	\tableofcontents
\end{frame}

\section{Wiederholung}
\subsection{Übungsblatt NR 2}
\begin{frame}
	\frametitle{Vorletztes \"Ubungsblatt}
	\begin{block}{Probleme}
		\begin{itemize}
			\item Formale Definitionen
			\item Vollständige Induktion
			\item induktiv $\approx$ rekursiv
		\end{itemize}
	\end{block}
\end{frame}

\subsection{Übungsblatt NR 3}
\begin{frame}
	\frametitle{Letztes \"Ubungsblatt}
	\begin{block}{Probleme}
		\begin{itemize}
			\item 11101 = -3
			\item $-2^x$
			\item Definition von Surjektivität ist kein Beweis
			\item schreibt nicht so Wörter wie Natürlich oder 						Selbstverständlich in Beweise
		\end{itemize}
	\end{block}
\end{frame}

\section{Prozessor}
\subsection{Aufbau}
\begin{frame}
	\frametitle{Aufbau}
	\begin{block}{}
		Was brauchen wir in einem Prozessor
	\end{block}
\end{frame}

\begin{frame}
	\frametitle{Aufbau}
	\begin{block}{}
		\begin{itemize}
			\item Register: Speicher für je ein Wort
			\item Rechenwerk (ALU): berechnet Arithmetische und Logische Funktionen 
			\item Steuerwerk: (Befehlsregister, Befehlszähler, Statusregister (Flags wie Übertrag, Minus..))
			\item Speicherwerk: organisiert Speicherzugriff
			\item Daten- und Adressbus: Transfer zwischen Bausteinen
		\end{itemize}
	\end{block}
\end{frame}

\section{Befehle}
\subsection{Befehlaufbau}
\begin{frame}
	\frametitle{Befehlaufbau}
	\begin{block}{Definition}
		Übermittlung einer Bitfolge als Befehl and den Prozessor\\
		0010 0000 0000 0000 0010 1010 \\
		= STV 10101\\
		= STV 42
	\end{block}
\end{frame}

\subsection{Befehle}
\begin{frame}
	\frametitle{ein paar Befehle}
	\begin{block}{Definition}
		\begin{itemize}
			\item LDC const: Lädt Konstante in Akku.
			\item LDV adr: Lädt Wert der Adresse in Akku.
			\item STV adr: Speichert Wert in Adresse .
			\item LDIV adr: Lädt Wert von der Adresse welche in der 				Adresse die mitgeliefert wurde gespeichert ist.
			\item ADD adr: Addiert den Wert an der Adresse mit dem 					Wert im Akku und speichert das Ergebnis im Akku
			\item EQL adr: wenn Akku und Wert an der Adresse übereinstimmen wird -1 in dem Akku geladen, sonst 0
			\item AND, OR, XOR 
		\end{itemize}
	\end{block}
\end{frame}

\subsection{Programmablauf}
\begin{frame}
	\frametitle{normaler Programmablauf}
	\begin{block}{Definition}
		Steuerregister hält Liste von Befehlen und speichert derzeitige Position. \\
		\begin{itemize}
			\item 001 LDV 41
			\item 010 ADD 15
			\item 011 STV 12
			\item 100 LDC 10
			\item ...
		\end{itemize}
		Der normale Ablauf ist eine Schrittweise abarbeitung der Befehle. Gibt es Situationen in denen wir mehr brauchen?
	\end{block}
\end{frame}

\begin{frame}
	\frametitle{Spr\"unge}
	\begin{block}{Definition}
		Für if/else, Schleifen, ... gibt es Sprungbefehle \\
		\begin{itemize}
			\item JMP adr: springt zu gegebener Adresse
			\item JMN adr: springt wenn Wert in Akku negative ist.
		\end{itemize}
		Wie baut man eine if/else Abfrage?
	\end{block}
\end{frame}

\begin{frame}
	\frametitle{Durchf\"uhrung in drei Phasen}
	\begin{block}{die drei Phasen}
		\begin{itemize}
			\item Holphase
			\item Decodierphase
			\item Ausführungsphase
		\end{itemize}
	\end{block}
\end{frame}

\section{Beispiele}
\subsection{Multiplizieren}
\begin{frame}
	\frametitle{3 mal 4}
	\begin{block}{Variablen Definitionen}
		var1: 3\\
		var2: 4\\
		i:\\
		result:
	\end{block}
\end{frame}

\begin{frame}
	\begin{block}{Code}
		\begin{tabular}{ll}
		 LDC 0 & Lädt die Konstante 0\\
		 STV result	& 0 $->$ Ergebnisvariable\\
		 STV i & 0 $->$ Schleifenvariable\\
		 start: LDV result & Lädt das Ergebnis\\
		 Add var2 & Addiert 4\\
		 STV result & Speichert erhöhtes Ergebnis\\
		 LDC 1 & Lädt Konstante 1\\
		 ADD i & increment Schleifenbariable\\
		 STV i & speichert das incrementierte i\\
		 EQL var1 & kontrolliert ob genug oft addiert wurde\\
		 JMN done & springe zum Ende wenn fertig\\
		 JMP start & springe zurück zum Anfang\\
		 done: HALT & beendet das Programm
		 \end{tabular}
	\end{block}
\end{frame}

\subsection{weitere Beispiele}
\begin{frame}
	\frametitle{Beispiele}
	\begin{block}{}
		\begin{itemize}
			\item if abfrage mit a = b = c\\
			\item geschachtelte for schleife
		\end{itemize}
	\end{block}
\end{frame}

\begin{frame}
	\begin{center}
		Fragen?
	\end{center}
\end{frame}

\begin{frame}
	\frametitle{Unnützes Wissen}
	\begin{center}
		Als Bibliotaphen  bezeichnet man Menschen, die ihre Bücher in 		Verstecken aufbewahren und nicht verleihen.
	\end{center}
\end{frame}
\end{document}