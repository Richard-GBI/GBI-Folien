\documentclass{beamer}
\usepackage[ngerman]{babel}
\usepackage[utf8]{inputenc}
\usepackage{graphicx}
\usepackage{amssymb}
\usepackage{amsmath}
\usepackage{hyperref}
\usetheme{Warsaw}
\usecolortheme{default}

\author{Richard Feistenauer}
\title{GBI-Tutorium 2}
\date{7.November.2014}

\begin{document}
\begin{frame}
	\titlepage
\end{frame}

\begin{frame}
	\frametitle {Inhaltsverzeichnis}
	\tableofcontents
\end{frame}

\section{Wiederholung}
\subsection{Übungsblatt}
\begin{frame}
	\frametitle{Letztes \"Ubungsblatt}
	\begin{block}{Probleme}
		\begin{itemize}
			\item 1.1 $\forall$ x $\in$ $\{\}$
			\item 1.3 0 teilt die 0
			\item 1.5 aus $=>$ folgt nicht $<=>$
		\end{itemize}
	\end{block}
\end{frame}

\section{Übersetzung}
\subsection{Zahlendarstellung}
\begin{frame}
	\frametitle{Zahlendarstellung}
	\begin{block}{Definition}
		Definiere num$_{10}$(x).\\
		\pause
		\hfill\\
		Sei Z$_{10}$ = $\{$0, 1, ..., 9$\}$, so definieren wir die 				Dezimaldarstellung von Zahlen so:\\
		\hfill\\
		Num$_{10}$($\epsilon$) = 0\\
		$\forall w \in$ Z$_{10}^*$ $ \forall x \in$ Z$_{10}$: Num				$_{10}$(wx) = 10 $\cdot$ Num$_{10}$(w) + num$_{10}$(x)
	\end{block}
\end{frame}

\begin{frame}
	\frametitle{Andere Zahlendarstellungen}
	\begin{block}{Beispiele}
		Man kann nun nicht nur Zahle des im Zahlensystem der Basis 10 		berechnen, sondern auch Zahler einer beliebigen Basis k.\\
		\hfill\\
		\begin{block}{Beispielaufgaben}
			\begin{itemize}
				\item Num$_{2}$(101) = \pause 5
				\item Num$_{5}$(431) = \pause 116
				\item Num$_{8}$(12) = \pause 10
			\end{itemize}
		\end{block}
	\end{block}
\end{frame}

\subsection{Zweierkomplement}
\begin{frame}
	\frametitle{Zweierkomplement}
	\begin{block}{Negative Zahlen}
		Negative Zahlen Binär darstellen ohne weiteres Zeichen? \\
		die Negation einer Positiven Zahl ist eine Negative Zahl
		
		\begin{itemize}
			\item ...
			\item 001 = 1
			\item 000 = 0
			\item 111 = -1
			\item 110 = -2
			\item ...
		\end{itemize}
	\end{block}
\end{frame}

\begin{frame}
	\frametitle{Zweierkomplement}
	\begin{block}{Negative Zahlen}
		In GBI ist $\mathbb{K}_l$ die Darstellung für 							Zweierkomplement\\
		$\mathbb{K}_5$ = $\{$-16,-15,..,-1,0,1,..,14,15$\}$
	\end{block}
\end{frame}

\subsection{Homomorphismen}
\begin{frame}
	\frametitle{Übersetzungen}
	\begin{block}{Übersetzungen}
		Wozu braucht man überhaupt Übersetzungen?\\
		\pause
		\begin{itemize}
			\item Lesbarkeit
			\item Kompression
			\item Verschlüsselung
			\item Fehlererkennung und Fehlerkorrektur
		\end{itemize}
	\end{block}
\end{frame}

\begin{frame}
	\frametitle{Homomorphismen}
	\begin{block}{Präfixe}
		Homomorphismus: Seien A und B zwei Alphabete.\\
		$h: A \rightarrow B$ ist ein Homomorphismus, wenn gilt:\\
		\begin{center}
			$h(\epsilon ) = \epsilon$\\
			$\forall w \in A^*: \forall x \in A: h(wx) = h(w)h(x)$
		\end{center}\pause
		\hfill\\	
		Präfixfreier Code: für keine zwei verschiedenen Symbole 				$x_1$, $x_2$ $\in$ A gilt: h($x_1$)
		ist ein Präfix von h($x_2$).\pause
		\\
		\hfill\\
		$\epsilon$-freier Homomorphismus		
	\end{block}
\end{frame}

\subsection{Huffman-Code}
\begin{frame}
	\frametitle{Huffman-Code}
	\begin{block}{Huffman}
		Der Huffman-Code ist ein Code, der unter allen präfixfreien 			Codes zu den kürzesten Codierungen führt.\\
		\hfill\\
		Wichtig ist dafür, dass wir die Anzahl gewisser Symbole 				unseres zu codierenden Textes kennen.\\
		\pause
		\begin{itemize}
			\item Für jedes zu kodierende Symbol erstellen wir einen 				Knoten, das das Symbol und seine Anzahl beinhaltet.
			\item Nun nehmen wir immer die zwei Knoten mit der 						kleinsten Anzahl, zählen die Anzahlen zusammen und 						erstellen einen Baum mit dem neu erstellten Knoten als 					Wurzel
			\item Wir beschriften alle Kanten, die nach rechts gehen 				mit 1 und alle nach links mit 0.
		\end{itemize}
	\end{block}
\end{frame}

\begin{frame}
	\frametitle{Beispielaufgaben}
	\begin{block}{\bf Wir haben acht Symbole a, b, c, d, e, f, g, h}
		\begin{itemize}
			\item Jedes Zeichen kommt einfach vor. Wie sieht der 					Huffman-Code aus?\\
			Wie lang ist die Codierung von edcbahfg?\pause
			\item a kommt einmal vor, b zweimal, c 4-mal, d 8-mal, e 				16-mal, f 32-mal,
			g 64-mal, h 128-mal. Erstelle einen Huffman Baum.
		\end{itemize}
	\end{block}
\end{frame}

\begin{frame}
	\frametitle{Block-Codierung}
	\begin{block}{Block-Codierung}
		Wie würdet ihr den Huffman-Code für das folgende Wort 					definieren:\\
		$aaaaaabbbbbbccccccddddddaaadddddd$
		\pause		
		\\
		\hfill\\
		Man kann natürlich nicht nur einzelne Symbole codieren, 				sondern auch Symbolblöcke.
	\end{block}
\end{frame}

\section{Übungen}
\subsection{Übung}
\begin{frame}
	\begin{center}
		Tafel
	\end{center}
\end{frame}

\begin{frame}
	\begin{center}
		Fragen?
	\end{center}
\end{frame}

\begin{frame}
	\frametitle{Unnützes Wissen}
	\begin{center}
		Jack Nicholson fand erst mit 37 Jahren heraus, dass seine 				Schwester in Wahrheit seine Mutter ist.
	\end{center}
\end{frame}
\end{document}