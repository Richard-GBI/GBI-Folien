\documentclass{beamer}
\usepackage[ngerman]{babel}
\usepackage[utf8]{inputenc}
\usepackage{graphicx}
\usepackage{amssymb}
\usepackage{amsmath}
\usepackage{hyperref}
\newcommand{\tief}[1]{\ensuremath{\mathrm{#1}}}
\usetheme{Warsaw}
\usecolortheme{default}
\author{Richard Feistenauer}
\title{5.GBI-Tutorium von Tutorium Nr. 31}
\date{18.November 2014}

\begin{document}
\begin {frame}
	\titlepage
\end {frame}

\begin {frame}
	\frametitle {Inhaltsverzeichnis}
	\tableofcontents
\end {frame}

\section{Wiederholung}
\begin{frame}
	\frametitle{Wiederholung}
	\begin{block}{Übungsblatt}
		\begin{itemize}
			\item Algorithmus für Huffmancode
			\item Sonstiges
			\begin{itemize}
				\item https://github.com/Richard-GBI
			\end{itemize}
		\end{itemize}
	\end{block}
\end{frame}

\section{Algorithmen}
\subsection{Definition}
\begin{frame}
	\frametitle{Definition}
	\begin{block}{Informell}
		Eine von einer Maschine abarbeitbare Anleitung, ein
		bestimmtes Problem zulösen
	\end{block}
	\begin{block}{Eigenschaften}
		\pause
		\begin{itemize}
			\item endliche Beschreibung
			\item elementare Anweisungen
			\item Determinismus
			\item zu \emph{endlicher} Eingabe wird \emph{endliche} 					Ausgabe erzeugt
			\item endliche viele Schritte
			\item funktioniert für beliebig große Eingaben
			\item Nachvollziehbarkeit/Verständlichkeit für jeden (der 			mit der Materie vertraut ist)
		\end{itemize}
	\end{block}
\end{frame}

\subsection{Pseudocode}
\begin{frame}
	\begin{block}{Pseudocode}
		\begin{tabular}[c]{ll}
			Zuweisung & x $\leftarrow$ a \\
			&\\
			for-Schleife & Schleifenvariable $\leftarrow$ Startwert 				to Endwert \\
			& do \\
			& \begin{quote} Schleifenrumpf \end{quote}\\
			& od \\
		\end{tabular}
	\end{block}
\end{frame}

\begin{frame}
	\frametitle{Übung}
	\begin{block}{Beispiel: Übungsblatt 3/08}
		Schreiben Sie einen Algorithmus auf, der folgendes leistet
		\begin{itemize}
			\item Als Eingaben erhält er ein Wort w:$G_n \rightarrow$ 			A und zwei Symbole x $\in$ A und y $\in$ A
			\item Am Ende soll eine Variable r den Wert 0 oder 1 					haben, und zwar soll gelten:\\	
			$r = \begin{cases} 1 &\mbox{falls irgendwo in w direkt 						hintereinander} \\
			& \mbox{erst x und dann y vorkommen}\\
			0 & \mbox{sonst}. \end{cases}$
		\end{itemize}
		Benutzen Sie zum Zugriff auf das i-te Symbol von w die 					Schreibweise w(i). Formulieren Sie den Algorithmus mit Hilfe 			einer for-Schleife.
	\end{block}
\end{frame}

\begin{frame}
	\frametitle{Übung}
	\begin{block}{Lösung}
		\begin{tabbing}
			$p \leftarrow 0$\\
			\textbf{for} \= $i \leftarrow0$ 			\textbf{to} $n-2$ \textbf{do}\\
			\>$p \leftarrow$
			$\begin{cases}$
			1 $&$ falls $w(i)=x \land w(i+1) = y\\ $
			p $&$ sonst
			$\end{cases}$\\
			\textbf{od}
		\end{tabbing}
	\end{block}
\end{frame}

\begin{frame}
	\frametitle{Nochmal eins}
	\begin{block}{Beispiel: GGT}
		Ein Pseudocode für den größten gemeinsamen Teiler.
	\end{block}
\end{frame}

\subsection{Schleifeninvariante}
\begin{frame}
	\frametitle{Schleifeninvariante}
	\begin{block}{Schleifeninvariante}
		\begin{itemize}
			\item arithmetische Aussage
			\item betrifft alle interessanten Parameter der Schleife
			\item gilt immer
			\begin{itemize}
				\item vor dem ersten Schleifendurchlauf
				\item nach jedem einzelnen Schleifendurchlauf
				\item nach Ende der Schleife
			\end{itemize}
		\end{itemize}
	\end{block}
	\begin{block}{Beweis}
		Der Beweis einer Schleifeninvariante \\
		ist häufig durch Vollständige Induktion möglich.
	\end{block}
\end{frame}

\begin{frame}
	\frametitle{Schleifeninvariante}
	\begin{block}{Aufgabe}
		S$_0$ $\leftarrow$ a \\
		Y$_0$ $\leftarrow$ b \\
		for i$\leftarrow$0 to b-1 \\
		do \\
		\begin{quote}
			S$_{i+1}$ $\leftarrow$ S$_i$ + 1 \\
			Y$_{i+1}$ $\leftarrow$ Y$_i$ - 1
		\end{quote}
		od \\
		Was tut dieser Algorithmus ?
	\end{block}
\end{frame}

\begin{frame}
	\frametitle{Schleifeninvariante}
	\begin{block}{Aufgabe}
		Wir wählen a = 6 und b = 4 \bigskip \\
		\begin{tabular}[c]{| c | c | c |}
			\hline
			& S$_i$ & Y$_i$ \\
			\hline
			\hline
			i = 0 & & \\
			\hline
			i = 1 & & \\
			\hline
			i = 2 & & \\
			\hline
			i = 3 & & \\
			\hline
			i = 4 & & \\
			\hline
		\end{tabular}
		\bigskip \\
		Was wäre eine Mögliche Schleifeninvariante?\\
		\pause
		$\forall i \in \tief{\mathbb{N}_0}: i \leq b \Rightarrow S_i 			+ Y_i = a + b$
	\end{block}
\end{frame}

\section{Fragen}
\begin{frame}
	\begin{center}
		Noch Fragen?
	\end{center}
\end{frame}

\begin{frame}
	\frametitle{Unnützes Wissen}
	\begin{center}
		In Nordsibirien ist es Brauch, dass verliebte Frauen ihren 				Angebeteten mit Feldschnecken bewerfen.
	\end{center}
\end{frame}
\end {document}